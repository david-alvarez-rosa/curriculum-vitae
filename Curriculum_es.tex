\PassOptionsToPackage{dvipsnames}{xcolor}
\documentclass[10pt, a4paper, ragged2e]{altacv}

\geometry{left = 1cm, right = 9cm, marginparwidth = 6.8cm, marginparsep = 1.2cm,
  top = 1.25cm, bottom = 1.25cm}

\usepackage[utf8]{inputenc}
\usepackage[T1]{fontenc}
\usepackage[default]{lato}
\usepackage[hidelinks]{hyperref}
\usepackage{changepage}

\definecolor{Mulberry}{HTML}{72243D}
\definecolor{SlateGrey}{HTML}{2E2E2E}
\definecolor{LightGrey}{HTML}{666666}
\colorlet{heading}{Sepia}
\colorlet{accent}{Mulberry}
\colorlet{emphasis}{SlateGrey}
\colorlet{body}{LightGrey}


\newcommand{\grade}[1]{%
  \begin{tikzpicture}
    \clip (1em-.4em,-.35em) rectangle (5em +.5em ,1em);
    \foreach \x in {1,2,...,5}{
        \path[{fill=body!30}] (\x em,0) circle (.35em); % backColor
    }
    \begin{scope}
    \clip (1em-.4em,-.35em) rectangle (#1em +.5em ,1em);
    \foreach \x in {1,2,...,5}{
        \path[{fill=accent}] (\x em,0) circle (.35em); % frontColor
    }
    \end{scope}
  \end{tikzpicture}%
}

\renewcommand{\cvskill}[2]{%
  \textcolor{emphasis}{\textbf{#1}}\hfill
  \grade{#2}\par
}


\begin{document}

\name{David Álvarez Rosa}
\tagline{Estudiante de Matemáticas e Ingeniería Industrial \vspace{.05cm}}
\photo{1cm}{blank}
\personalinfo{%
  \email{\href{mailto:david@alvarezrosa.com}{\hspace{.025cm}david@alvarezrosa.com}}
  \phone{\href{tel:+34647133930}{+34 647 13 39 30 \hspace{.925cm}}}
  \mailaddress{
    \href{https://www.google.com/maps/place/Calle+Agrupación+Olaz,+16,+31620+Olaz}
    {C/ Agrupación Olaz nº 16, Bajo}}
  \location{\href{https://es.wikipedia.org/wiki/Olaz}{Olaz, Navarra}}
  \homepage{\href{https://david.alvarezrosa.com}{david.alvarezrosa.com \hspace{.085cm}}}
  \printinfo{\faGitlab}{\href{https://gitlab.com/DavidAlvarez}
    {gitlab.com/DavidAlvarez}}
  \printinfo{\faBirthdayCake}{ Octubre 10, 1998}
}

\begin{adjustwidth}{0pt}{-10pt}
  \begin{fullwidth}
    \makecvheader
  \end{fullwidth}
\end{adjustwidth}

\medskip
\cvsection[page1sidebar_es]{Educación}

\cvevent{\href{https://www.upc.edu/es/grados/matematicas-barcelona-fme}
  {Grado en Matemáticas} \hfill \small{240
    \href{https://es.wikipedia.org/wiki/European_Credit_Transfer_and_Accumulation_System}
    {ECTS}}}
{\href{https://upc.edu}{Universidad Politécnica de Cataluña} --
  \href{https://fme.upc.edu}{FME}}
{Septiembre 2016 -- Presente}{Barcelona, Cataluña}
\begin{itemize}
  \item Cursando 4º curso actualmente.
\end{itemize}
Robusta base teórica matemática y sólidos conocimientos en sus aplicaciones
(algoritmos, computación).

\divider

\cvevent{\href{https://www.upc.edu/es/grados/ingenieria-en-tecnologias-industriales-barcelona-etseib}
  {Grado en Ingeniería en Tecnologías Industriales} \hfill \small{240
    \href{https://es.wikipedia.org/wiki/European_Credit_Transfer_and_Accumulation_System}
    {ECTS}}}
{\href{https://upc.edu}{Universidad Politécnica de Cataluña} --
\href{https://etseib.upc.edu}{ETSEIB}}
{Septiembre 2016 -- Presente}{Barcelona, Cataluña}
\begin{itemize}
  \item Cursando 4º curso actualmente.
\end{itemize}
Visión multidisciplinar e integradora de las ingenierías industriales.
Adquiridos conocimientos y habilidades imprescindibles para el desarrollo
tecnológico futuro.

\divider

\cvevent{Scientific and Technological Baccalaureate \hfill \small{2 years}}
{\href{https://www.irabia-izaga.org}{Irabia-Izaga school}}
{Septiembre 2014 -- June 2016}{Burlada, Navarra}
\begin{itemize}
  \item Final grade: 9,47/10.
  \item University access exam grade
  (\href{https://en.wikipedia.org/wiki/Selectividad}{\textit{Selectividad}}):
  12,76/14.
\end{itemize}

\divider

\cvevent{Compulsory Secondary Education \hfill \small{4 years}}
{\href{https://www.irabia-izaga.org}{Irabia-Izaga school}}
{Septiembre 2010 -- June 2014}{Burlada, Navarra}

\medskip
\cvsection{Courses}

\cvevent{\href{https://es.wikipedia.org/wiki/Teoría_de_juegos}
  {Game theory}
  \hfill \small{20 hours}}{\href{https://upc.edu}
  {Universidad Politécnica de Cataluña} -- \href{https://cfis.upc.edu}{CFIS}}
{April 2019}{Barcelona, Cataluña}
Game theory is the study of mathematical models of strategic interaction among
rational decision-makers. It has applications in fields such as economics, logic
and computer science.

\vspace{.1cm}
\divider
\vspace{.15cm}

\cvevent{\href{https://sites.google.com/view/dlcfis2019/home}
  {Introduction to Machine Learning \& Deep Learning}
  \hfill \small{20 hours}}
{\href{https://upc.edu}{Universidad Politécnica de Cataluña} --
  \href{https://cfis.upc.edu}{CFIS}}
{January 2019}{Barcelona, Cataluña}
\begin{itemize}
  \item Basic principles of machine learning and classical methods.
  \item Introduction to deep learning from both an algorithmic and computational
  point of view.
  \item Study of its applications to reinforced learning and the analysis of
  multimedia content.
\end{itemize}


\clearpage

\cvsection[page2sidebar_es]{Projects}

\cvevent{\href{https://driverless.upc.edu}{Driverless -- Motorsport}
  \hfill \small{20 hours/week}}
{\href{https://upc.edu}{Universidad Politécnica de Cataluña} --
\href{https://etseib.upc.edu}{ETSEIB}}
{Septiembre 2019 - Presente}{}
I'm part of the Perception section of Driverless UPC team, which is a team
formed by undergraduate engineers in charge of the designing, manufacturing and
testing of a fully autonomous car that will participate in national and
international competitions between universities.

\divider

\cvevent{\href{https://alvarezrosa.com/tres-en-raya}{Robotic Arm -- Tic-tac-toe}
  \hfill \small{75 hours}}
{\href{https://upc.edu}{Universidad Politécnica de Cataluña} --
\href{https://etseib.upc.edu}{ETSEIB}}
{January 2019 - June 2019}{}
Creation of a virtual animation of a robotic arm intelligently playing the game
of Tic-tac-toe.

\divider

\cvevent{Email server \hfill \small{15 hours}}
{Personal project}{August 2018 - Septiembre 2018}{}
\begin{itemize}
  \item Configuration of a modern and secure personal email server based on
  free software (\href{http://www.postfix.org/}{Postfix} and
  \href{https://www.dovecot.org/}{Dovecot}).
  \item Supports opportunistic
  \href{https://en.wikipedia.org/wiki/Transport_Layer_Security}{TLS} and
  \href{https://en.wikipedia.org/wiki/Internet_Message_Access_Protocol}
  {IMAP} access.
\end{itemize}

\divider

\cvevent{\href{http://david.alvarezrosa.com:5000}{Study Schedule}
  \hfill \small{60 hours}}
{\href{https://upc.edu}{Polytechnic University of Cataluña} --
\href{https://etseib.upc.edu}{ETSEIB}}
{January 2018 - June 2018}{}
Creation of a customized study schedule generator adapted to students, through
data analysis.


\medskip

\cvsection{Interests}

\wheelchart{1.5cm}{0.5cm}{%
  6/8em/accent!30/Family and \\ friends,
  3/8em/accent!40/Automation,
  8/8em/accent!60/Robotics,
  2/10em/accent/Sport,
  5/6em/accent!20/Data analysis
}

\end{document}
