\PassOptionsToPackage{dvipsnames}{xcolor}
\documentclass[10pt, a4paper, ragged2e]{altacv}

\geometry{left = 1cm, right = 9cm, marginparwidth = 6.8cm, marginparsep = 1.2cm,
  top = 1.25cm, bottom = 1.25cm}

\usepackage[utf8]{inputenc}
\usepackage[T1]{fontenc}
\usepackage[default]{lato}
\usepackage[hidelinks]{hyperref}
\usepackage{changepage}
\usepackage[spanish]{babel}


\definecolor{Mulberry}{HTML}{72243D}
\definecolor{SlateGrey}{HTML}{2E2E2E}
\definecolor{LightGrey}{HTML}{666666}
\colorlet{heading}{Sepia}
\colorlet{accent}{Mulberry}
\colorlet{emphasis}{SlateGrey}
\colorlet{body}{LightGrey}


\newcommand{\grade}[1]{%
  \begin{tikzpicture}
    \clip (1em-.4em,-.35em) rectangle (5em +.5em ,1em);
    \foreach \x in {1,2,...,5}{
        \path[{fill=body!30}] (\x em,0) circle (.35em); % backColor
    }
    \begin{scope}
    \clip (1em-.4em,-.35em) rectangle (#1em +.5em ,1em);
    \foreach \x in {1,2,...,5}{
        \path[{fill=accent}] (\x em,0) circle (.35em); % frontColor
    }
    \end{scope}
  \end{tikzpicture}%
}

\renewcommand{\cvskill}[2]{%
  \textcolor{emphasis}{\textbf{#1}}\hfill
  \grade{#2}\par
}


\begin{document}

\name{David Álvarez Rosa}
\tagline{Estudiante de Matemáticas e Ingeniería Industrial \vspace{.05cm}}
\photo{1cm}{blank}
\personalinfo{%
  \email{\href{mailto:david@alvarezrosa.com}{\hspace{.025cm}david@alvarezrosa.com}}
  \phone{\href{tel:+34647133930}{+34 647 13 39 30 \hspace{.925cm}}}
  \mailaddress{
    \href{https://www.google.com/maps/place/Calle+Agrupación+Olaz,+16,+31620+Olaz}
    {C/ Agrupación Olaz nº 16, Bajo}}
  \location{\href{https://es.wikipedia.org/wiki/Olaz}{Olaz, Navarra}}
  \homepage{\href{https://david.alvarezrosa.com}{david.alvarezrosa.com \hspace{.085cm}}}
  \printinfo{\faGitlab}{\href{https://gitlab.com/DavidAlvarez}
    {gitlab.com/DavidAlvarez}}
  \printinfo{\faLinkedin}{\href{https://www.linkedin.com/in/david-alvarez-rosa/}
    { \hspace{.05cm}david-alvarez-rosa}}
  \printinfo{\faBirthdayCake}{ Octubre 10, 1998}
}

\begin{adjustwidth}{0pt}{-10pt}
  \begin{fullwidth}
    \makecvheader
  \end{fullwidth}
\end{adjustwidth}

\medskip
\cvsection[page1sidebar_es]{Educación \hfill \Large \faGraduationCap}

\cvevent{\href{https://www.upc.edu/es/grados/matematicas-barcelona-fme}
  {Grado en Matemáticas} \hfill \small{240
    \href{https://es.wikipedia.org/wiki/European_Credit_Transfer_and_Accumulation_System}
    {ECTS}}}
{\href{https://upc.edu}{Universidad Politécnica de Cataluña} --
  \href{https://fme.upc.edu}{FME}}
{Septiembre 2016 -- Presente}{Barcelona, Cataluña}
\begin{itemize}
  \item \textbf{Asignaturas} destacadas: Álgebra Lineal, Cálculo, Programación
  Matemática, Algoritmia, Álgebra Abstracta, Geometría, Análisis, Ecuaciones
  Diferenciales, Probabilidad y Estadística.
\end{itemize}
Un grado riguroso y técnico con una robusta base teórica matemática y sólidos
conocimientos en sus aplicaciones (algoritmos, computación).

\divider

\cvevent{\href{https://www.upc.edu/es/grados/ingenieria-en-tecnologias-industriales-barcelona-etseib}
  {Grado en Ingeniería en Tecnologías Industriales} \hfill \small{240
    \href{https://es.wikipedia.org/wiki/European_Credit_Transfer_and_Accumulation_System}
    {ECTS}}}
{\href{https://upc.edu}{Universidad Politécnica de Cataluña} --
\href{https://etseib.upc.edu}{ETSEIB}}
{Septiembre 2016 -- Presente}{Barcelona, Cataluña}
\begin{itemize}
  \item \textbf{Asignaturas} destacadas: Mecánica, Termodinámica,
  Electromagnetismo, Electrotecnia, Mecánica de Fluidos, Materiales,
  Electrónica y Control.
\end{itemize}
Visión multidisciplinar e integradora de las ingenierías industriales.
Adquiridos conocimientos y habilidades imprescindibles para el desarrollo
tecnológico futuro.

\divider

\cvevent{Bachillerato Científico y Tecnológico \hfill \small{2 años}}
{\href{https://www.irabia-izaga.org}{Colegio Irabia-Izaga}}
{Septiembre 2014 -- Junio 2016}{Burlada, Navarra}
\begin{itemize}
  \item Nota final: 9,47/10.
  \item Nota prueba de acceso a la universidad
  (\href{https://es.wikipedia.org/wiki/Selectividad_(examen)}{Selectividad}):
  12,76/14.
\end{itemize}

\divider

\cvevent{Educación Secundaria Obligatoria \hfill \small{4 años}}
{\href{https://www.irabia-izaga.org}{Colegio Irabia-Izaga}}
{Septiembre 2010 -- Junio 2014}{Burlada, Navarra}

\medskip
\cvsection{Cursos \hfill \Large \faBook}

\cvevent{\href{https://es.wikipedia.org/wiki/Teoría_de_juegos}
  {Teoría de Juegos}
  \hfill \small{20 horas}}{\href{https://upc.edu}
  {Universidad Politécnica de Cataluña} -- \href{https://cfis.upc.edu}{CFIS}}
{Abril 2019}{Barcelona, Cataluña}
La teoría de juegos consiste en el estudio de \textbf{modelos matemáticos} de
interacción estratégica entre tomadores de decisiones \textbf{racionales}. Tiene
aplicaciones en campos como la economía, lógica y computación.

\vspace{.1cm}
\divider
\vspace{.15cm}

\cvevent{\href{https://sites.google.com/view/dlcfis2019/home}
  {\textit{Introduction to Machine Learning \& Deep Learning\footnotemark}}
  \hfill \small{20 horas}}
{\href{https://upc.edu}{Universidad Politécnica de Cataluña} --
  \href{https://cfis.upc.edu}{CFIS}}
{Enero 2019}{Barcelona, Cataluña}
\begin{itemize}
  \item Principios básicos de \textit{machine learning} y métodos clásicos.
  \item Introducción al \textit{deep learning} tanto desde un punto de vista
  algorítmico como computacional.
  \item Estudio de sus aplicaciones al aprendizaje reforzado y al análisis de
  contenido multimedia.
\end{itemize}


\clearpage

\cvsection[page2sidebar_es]{Proyectos \hfill \Large \faCogs}

\cvevent{\href{https://driverless.upc.edu}{\textit{Driverless -- Motorsport}}
  \hfill \small{20 horas/semana -- 6 meses}}
{\href{https://upc.edu}{Universidad Politécnica de Cataluña} --
\href{https://etseib.upc.edu}{ETSEIB}}
{Septiembre 2019 - Febrero 2020}{}
He formado parte de la sección de \textbf{Percepción} del equipo
\href{https://driverless.upc.edu}{\textit{Driverless} UPC}, que es un equipo
formado por estudiantes de ingeniería encargados del diseño, fabricación y
pruebas de un coche (eléctrico) que puede conducir de manera \textbf{completamente
autónoma}, y que participará en competiciones nacionales e internacionales
interuniversitarias.

\divider

\cvevent{\href{https://alvarezrosa.com/tres-en-raya}
  {Brazo robótico -- Tres en Raya\footnotemark}
  \hfill \small{75 horas}}
{\href{https://upc.edu}{Universidad Politécnica de Cataluña} --
\href{https://etseib.upc.edu}{ETSEIB}}
{Enero 2019 - Junio 2019}{}
Creación de una animación virtual de un \textbf{brazo robótico} jugando de
manera inteligente al juego del Tres en Raya y \textbf{nunca pierda}. El código
de este proyecto puede ser consultado en mi
\href{https://gitlab.com/DavidAlvarez}{página de Gitlab\footnotemark}.

\divider

\cvevent{Servidor de Correo Electrónico \hfill \small{15 horas}}
{Proyecto personal}{Agosto 2018 - Septiembre 2018}{}
\begin{itemize}
  \item Configuración de un moderno y seguro servidor de correo electrónico
  personal basado en \textit{software} libre
  (\href{http://www.postfix.org/}{Postfix} y
  \href{https://www.dovecot.org/}{Dovecot}).
  \item Soporta \href{https://es.wikipedia.org/wiki/Transport_Layer_Security}{TLS}
  oportunista y acceso
  \href{https://es.wikipedia.org/wiki/Protocolo_de_acceso_a_mensajes_de_Internet}
  {IMAP}.
\end{itemize}

\divider

\cvevent{\href{http://david.alvarezrosa.com:5000}
  {Horario de estudio\footnotemark}
  \hfill \small{60 horas}}
{\href{https://upc.edu}{Universidad Politécnica de Cataluña} --
\href{https://etseib.upc.edu}{ETSEIB}}
{Enero 2018 - Junio 2018}{}
Creación de un generador de horarios de estudios personalizados adaptado a
estudiantes, mediante \textbf{análisis de
  datos}: \textit{clustering/k-nearest-neighbours}. En particular, basándonos en
notas anteriores, rendimiento deseado y restricciones de horario, fuimos capaces
de crear el mejor horario de estudio posible para \textbf{maximizar} el
\textbf{rendimiento} del alumno.


\medskip

\cvsection{Intereses \hfill \Large \faSearch}

\wheelchart{1.5cm}{0.5cm}{%
  6/8em/accent!30/Familia y \\ amigos,
  3/8em/accent!40/Automatización,
  8/8em/accent!60/Robótica,
  2/10em/accent/Deporte,
  5/6em/accent!20/Análisis de datos
}

\end{document}
