\PassOptionsToPackage{dvipsnames}{xcolor}
\documentclass[10pt, a4paper, ragged2e]{altacv}

\geometry{left = 1cm, right = 9cm, marginparwidth = 6.8cm, marginparsep = 1.2cm,
  top = 1.25cm, bottom = 1.25cm}

\usepackage[utf8]{inputenc}
\usepackage[T1]{fontenc}
\usepackage[default]{lato}
\usepackage[hidelinks]{hyperref}
\usepackage{changepage}

\definecolor{Mulberry}{HTML}{72243D}
\definecolor{SlateGrey}{HTML}{2E2E2E}
\definecolor{LightGrey}{HTML}{666666}
\colorlet{heading}{Sepia}
\colorlet{accent}{Mulberry}
\colorlet{emphasis}{SlateGrey}
\colorlet{body}{LightGrey}


\newcommand{\grade}[1]{%
  \begin{tikzpicture}
    \clip (1em-.4em,-.35em) rectangle (5em +.5em ,1em);
    \foreach \x in {1,2,...,5}{
        \path[{fill=body!30}] (\x em,0) circle (.35em); % backColor
    }
    \begin{scope}
    \clip (1em-.4em,-.35em) rectangle (#1em +.5em ,1em);
    \foreach \x in {1,2,...,5}{
        \path[{fill=accent}] (\x em,0) circle (.35em); % frontColor
    }
    \end{scope}
  \end{tikzpicture}%
}

\renewcommand{\cvskill}[2]{%
  \textcolor{emphasis}{\textbf{#1}}\hfill
  \grade{#2}\par
}


\begin{document}

\name{David Álvarez Rosa}
\tagline{Mathematics and Industrial Engineering student \vspace{.05cm}}
\photo{1cm}{blank}
\personalinfo{%
  \email{\href{mailto:david@alvarezrosa.com}{\hspace{.025cm}david@alvarezrosa.com}}
  \phone{\href{tel:+34647133930}{+34 647 13 39 30 \hspace{.925cm}}}
  \mailaddress{
    \href{https://www.google.com/maps/place/Calle+Agrupación+Olaz,+16,+31620+Olaz}
    {C/ Agrupación Olaz nº 16, Bajo}}
  \location{\href{https://es.wikipedia.org/wiki/Olaz}{Olaz, Navarre}}
  \homepage{\href{https://david.alvarezrosa.com}{david.alvarezrosa.com \hspace{.085cm}}}
  \printinfo{\faGitlab}{\href{https://gitlab.com/DavidAlvarez}
    {gitlab.com/DavidAlvarez}}
  \printinfo{\faBirthdayCake}{ October 10, 1998}
}

\begin{adjustwidth}{0pt}{-10pt}
  \begin{fullwidth}
    \makecvheader
  \end{fullwidth}
\end{adjustwidth}

\medskip
\cvsection[page1sidebar]{Education}

\cvevent{\href{https://www.upc.edu/en/bachelors/mathematics-barcelona-fme}
  {Degree in Mathematics} \hfill \small{240
    \href{https://en.wikipedia.org/wiki/European_Credit_Transfer_and_Accumulation_System}
    {ECTS}}}
{\href{https://www.upc.edu/en}{Polytechnic University of Catalonia} --
  \href{https://fme.upc.edu/en}{FME}}
{September 2016 -- Present}{Barcelona, Catalonia}
\begin{itemize}
  \item Currently in fourth grade.
\end{itemize}
Rigorous course with a robust mathematical base and providing a solid knowledge
in its applications (algorithms, computing).

\divider

\cvevent{\href{https://www.upc.edu/en/bachelors/industrial-technology-engineering-barcelona-etseib}
  {Degree in Industrial Technology Engineering} \hfill \small{240
    \href{https://en.wikipedia.org/wiki/European_Credit_Transfer_and_Accumulation_System}
    {ECTS}}}
{\href{https://www.upc.edu/en}{Polytechnic University of Catalonia} --
\href{https://etseib.upc.edu/en}{ETSEIB}}
{September 2016 -- Present}{Barcelona, Catalonia}
\begin{itemize}
  \item Currently in fourth grade.
\end{itemize}
Multidisciplinary and integrative vision of industrial engineering. Acquired
knowledge and skills essential for future technological development.

\divider

\cvevent{Scientific and Technological Baccalaureate \hfill \small{2 years}}
{\href{https://www.irabia-izaga.org}{Irabia-Izaga school}}
{September 2014 -- June 2016}{Burlada, Navarre}
\begin{itemize}
  \item Final grade: 9,47/10.
  \item University access exam grade
  (\href{https://en.wikipedia.org/wiki/Selectividad}{\textit{Selectividad}}):
  12,76/14.
\end{itemize}

\divider

\cvevent{Compulsory Secondary Education \hfill \small{4 years}}
{\href{https://www.irabia-izaga.org}{Irabia-Izaga school}}
{September 2010 -- June 2014}{Burlada, Navarre}

\medskip
\cvsection{Courses}

\cvevent{\href{https://es.wikipedia.org/wiki/Teoría_de_juegos}
  {Game theory}
  \hfill \small{20 hours}}{\href{https://www.upc.edu/en}
  {Polytechnic University of Catalonia} -- \href{https://cfis.upc.edu}{CFIS}}
{April 2019}{Barcelona, Catalonia}
Game theory is the study of mathematical models of strategic interaction among
rational decision-makers. It has applications in fields such as economics, logic
and computer science.

\vspace{.1cm}
\divider
\vspace{.15cm}

\cvevent{\href{https://sites.google.com/view/dlcfis2019/home}
  {Introduction to Machine Learning \& Deep Learning}
  \hfill \small{20 hours}}
{\href{https://www.upc.edu/en}{Polytechnic University of Catalonia} --
  \href{https://cfis.upc.edu}{CFIS}}
{January 2019}{Barcelona, Catalonia}
\begin{itemize}
  \item Basic principles of machine learning and classical methods.
  \item Introduction to deep learning from both an algorithmic and computational
  point of view.
  \item Study of its applications to reinforced learning and the analysis of
  multimedia content.
\end{itemize}


\clearpage

\cvsection[page2sidebar]{Projects}

\cvevent{\href{https://driverless.upc.edu}{Driverless -- Motorsport}
  \hfill \small{20 hours/week}}
{\href{https://www.upc.edu/en}{Polytechnic University of Catalonia} --
\href{https://etseib.upc.edu/en}{ETSEIB}}
{September 2019 - Present}{}
I'm part of the Perception section of Driverless UPC team, which is a team
formed by undergraduate engineers in charge of the designing, manufacturing and
testing of a fully autonomous car that will participate in national and
international competitions between universities.

\divider

\cvevent{\href{https://alvarezrosa.com/tres-en-raya}{Robotic Arm -- Tic-tac-toe}
  \hfill \small{75 hours}}
{\href{https://www.upc.edu/en}{Polytechnic University of Catalonia} --
\href{https://etseib.upc.edu/en}{ETSEIB}}
{January 2019 - June 2019}{}
Creation of a virtual animation of a robotic arm intelligently playing the game
of Tic-tac-toe.

\divider

\cvevent{Email server \hfill \small{15 hours}}
{Personal project}{August 2018 - September 2018}{}
\begin{itemize}
  \item Configuration of a modern and secure personal email server based on
  free software (\href{http://www.postfix.org/}{Postfix} and
  \href{https://www.dovecot.org/}{Dovecot}).
  \item Supports opportunistic
  \href{https://en.wikipedia.org/wiki/Transport_Layer_Security}{TLS} and
  \href{https://en.wikipedia.org/wiki/Internet_Message_Access_Protocol}
  {IMAP} access.
\end{itemize}

\divider

\cvevent{\href{http://david.alvarezrosa.com:5000}{Study Schedule}
  \hfill \small{60 hours}}
{\href{https://www.upc.edu/en}{Polytechnic University of Catalonia} --
\href{https://etseib.upc.edu/en}{ETSEIB}}
{January 2018 - June 2018}{}
Creation of a customized study schedule generator adapted to students, through
data analysis.


\medskip

\cvsection{Interests}

\wheelchart{1.5cm}{0.5cm}{%
  6/8em/accent!30/Family and \\ friends,
  3/8em/accent!40/Automation,
  8/8em/accent!60/Robotics,
  2/10em/accent/Sport,
  5/6em/accent!20/Data analysis
}

\end{document}
